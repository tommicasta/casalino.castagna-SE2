\subsubsection{Scale Drivers}

The scale drivers are five distinct factors used to translate directly in the effort estimation process specific characteristics that have a huge relevance and impact on the project development.

To each scale driver can be assigned a value raging from \textit{very low} to \textit{very high}. The traslation in the effort estimation computation is performed associating to each scale driver's value a decimal value obtained through statistical analysis.

These five scale drivers are:

\begin{itemize}
	\item \textbf{Precedentedness}: this factor measures the degree of familiarity that the development team has regarding the project under analysis.

Our experience with this kind of projects is really limited, therefore a \textit{Low} value is assigned.
	\item \textbf{Development flexibility}: this factor reflects the degree of flexibility allowed for the development of the project. This factor is computed taking in consideration the presence of pre-enstablished requirements and software conformance with external interface specification.

Our projects involves some pre-defined requirements but no external interface specifications are required, hence the value of this scale driver is set to \textit{nominal}.
	\item \textbf{Architecture/Risk resolution}: scale factor measuring the quality of the risk management plan and scheduling/budget compatibility with the latter.

The risk management plan defined for this project embrace and define a recovery plan for almost every predictable risks, respecting the scheduling and the budget defined. \textit{High} value is assigned.
	\item \textbf{Team cohesion}: scale factor measuring reflecting the degree of cooperation among the team members.
Our team had divergence on project choices several time that brought a slow down in the project development. A \textit{Nominal} value is assigned.

	\item \textbf{Process maturity}: factor measuring the degree of maturity reached by the organization's development processes.
Since this is our first project, is quite hard to assess the maturity of our approach. Hence, even if the applied processes to the development could be typical of a level-3 organization, we set the factor's value to \textit{level 2}, typical of processes designed for specific projects. 
\end{itemize}

Refering to the scale drivers table, the following table is produced

\begin{table}[h!]
        \centering
        \begin{tabular}{ | c | c | c |}
                \hline
                \textbf{EO} & \textbf{Complexity} & \textbf{Weigth} \\
                \hline\
                Precedentedness & Low & 4.96 \\
                Development flexibility & Nominal & 3.04 \\
                Architecture/Risk resolution & High & 1.41 \\
                Team cohesion & Nominal & 3.29 \\
                Process maturity & Level 2 & 4.68 \\
                \hline
                Total value & \multicolumn{2}{c |}{17.38} \\
                \hline
        \end{tabular}
\end{table}

The \textit{total scale factor E} has a value computable by mean of the following equation

\begin{displaymath}
	E = B + 0.01 * \sum_{f} (SF_f) = 1,0838
\end{displaymath}

with $B = 0.91$ and $f$ representing one of the scale drivers.



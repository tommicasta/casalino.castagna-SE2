\subsubsection{Size evaluation}

In order to evaluate the approximated size through the function point method, the equation \ref{eq:UFP} is applied to the factors analyzed in the previous sections:

\begin{table}[h!]
        \centering
        \begin{tabular}{|c|c|}
	\hline
	\textbf{Function type} & \textbf{FPF} \\
	\hline
        ILF & 63 \\
	EIF & 20 \\
	EI & 71 \\
	EQ & 36 \\
	EO & 6 \\
	\hline
	\textbf{Total} & 196 \\
	\hline
	\end{tabular}
        \caption{UFP table}
        \label{table:UFPT}
\end{table}

Considering instead the \textbf{SLOC} as a project's size estimator, we can exploit the \textbf{gearing factors} to translate the UFP value in the corresponding SLOC estimation.

To give much flexibility as possible to the project, the gearing factors used to estimate the lower bound and the upper bound are calculated as the
arithmetical average between the gearing factors of J2EE and .NET, the two most widespread technologies.

\begin{table}[h!]
        \centering
        \begin{tabular}{|c|c|c|}
	\hline
	\textbf{Technology} & \textbf{Average} & \textbf{High} \\
	\hline
        J2EE & 46 & 67 \\
        .NET & 57 & 60 \\
	\hline
        \textbf{Average} & 52 & 64 \\
	\hline
        \end{tabular}
        \caption{UFP table}
        \label{table:UFPT}
\end{table}

Lower bound in KSLOC

\begin{displaymath}
	SLOC =  52 * 196 = 10.192
\end{displaymath}

Upper bound in KSLOC

\begin{displaymath}
	SLOC = 64 * 196 = 12.544
\end{displaymath}

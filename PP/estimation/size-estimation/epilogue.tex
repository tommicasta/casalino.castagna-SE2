\subsubsection{Size evaluation}

In order to evaluate the approximated size through the function point method, the equation \ref{eq:UFP} is applied to the factors analyzed in the previous sections:

\begin{table}[h!]
        \centering
        \begin{tabular}{|c|c|}
	\hline
	\textbf{Function type} & \textbf{FPF} \\
	\hline
        ILF & 63 \\
	EIF & 20 \\
	EI & 71 \\
	EQ & 36 \\
	EO & 6 \\
	\hline
	\textbf{Total} & 196 \\
	\hline
	\end{tabular}
        \caption{UFP table}
        \label{table:UFPT}
\end{table}

Considering instead the \textbf{SLOC} as a project's size estimator, we can exploit the \textbf{gearing factors} to translate the UFP value in the corresponding SLOC estimation.

For each gearing factor exist different values, corresponding the minimum, maximum and average number of source code lines for a single function point. 

In order to give a conservative estimation, we decide to assess the project's size using the average value and the maximum value to compute, respectively, its lower bound and the upper bound.

To give as much flexibility as possible to the project's implementation, without depending on specific technologies, languages or platforms, the gearing factors used to estimate the lower bound and the upper bound are calculated as the
arithmetical average between the gearing factors of J2EE and .NET, the two most widespread technologies.

\begin{table}[h!]
        \centering
        \begin{tabular}{|c|c|c|}
	\hline
	\textbf{Technology} & \textbf{Average} & \textbf{High} \\
	\hline
        J2EE & 46 & 67 \\
        .NET & 57 & 60 \\
	\hline
        \textbf{Average} & 52 & 64 \\
	\hline
        \end{tabular}
        \caption{UFP table}
        \label{table:UFPT}
\end{table}

Lower bound in KSLOC

\begin{displaymath}
	KSLOC^- =  52 * 196 = 10.192
\end{displaymath}

Upper bound in KSLOC

\begin{displaymath}
	KSLOC^+ = 64 * 196 = 12.544
\end{displaymath}

As explained in the introductory paragraph to this section, the size estimation effort is based on the estimation of the so-called \textit{Function Points}.

Function points are a statistical method of estimate the size of a software project evaluating the different functionalities provided by the software product in exam.

According to this approach, functionalities are divided into 5 \textbf{function types}, or categories:

\begin{itemize}
	\item \textbf{Internal Logical File}: a user identifiable group of logically related data that resides entirely within the application boundary, usually mantained by mean of external inputs.
	\item \textbf{External Interface File}: a user identifiable group of logically related data used by the application to provide part of its functionalities. The data resides entirely outside the application boundary. 
	\item \textbf{External Input}: elementary process in which data crosses the boundary from outside to inside. The data contained in the external input may be used to maintain one or more internal logical files.
	\item \textbf{External Output}: elementary process in which data generated by ILFs elaborations passes across the boundary from inside to outside.
	\item \textbf{External Inquiry}: elementary process with both input and output components that result in data retrieval from one or more internal logical files and external interface files, without their modification.
\end{itemize}

For each of these categories, a \textbf{weight} is associated. These weights are statistically determinated and vary according to the \textbf{complexity} of the function type.

The complexity of a function type can be derivided consulting the related \textbf{rating tables}.

In order to retrieve the number of function points assigned to the software product given a specific function type, a simple equation is applied:

\begin{equation}
	\label{eq:PFP}
	%% add a subtitle for this equation
	PFP_t = N_t * FP_t
\end{equation}

\begin{displaymath}
	t \in T = {\{ILF, EIG, EI, EO, EQ\}}
\end{displaymath}

This equation \eqref{eq:PFP} returns the \textbf{partial function points} obtained by multiplying the number of functionalities of a certain catergory for the weight associated to that category.

\begin{equation}
	\label{eq:UFP}
	UFP = \sum_{t} FPF_t, t \in T
\end{equation}

The total number of function points assigned to the whole project is computed by the \textbf{Unadjusted Function Points} equation \eqref{eq:UFP}, which simply compute the sum of each PFP previously calculated.

\section{ALGORITHM DESIGN}

This sections present some of the main algorithms used in the application. Those algorithms are written
in a pseudocode resembling the java syntax.

\subsection{Available car search}
The following algorithm searches all the available cars in the area shown in the map, if a car
is within the visible area of the map, this function will place it on the map in the correct position
so that the user can possibly reserve it.
% Referring to the Google Maps API this function should be
% added as a listener either to the "bounds\textunderscore changed" event and to the map generation event,
% so that the user can drag the map and still see the available cars.

\begin{algorithm}
\begin{algorithmic}[1]
  \caption{\label{alg:stdreqhandler} Car Search Handling Algorithm}

  \Function{nearbyAvailableCarSearch}{}
    \State $ bounds = \textit{map.getBounds()} $
    \State $ availableCars = \Call {getAvailableCars}{bounds} $
    \State $ i = \textit{0}$
    \For {$ i < \textit{availableCars.size()} $}
      \State \textit{map.addCar(
        \State position: availableCars[i].getLatLng(),
        \State title: availableCars[i].getId(),
        \State details: availableCars[i].getDetails()
      \State )}
      \State $ i++ $
    \EndFor
    \State \textbf{return}
  \EndFunction

\end{algorithmic}
\end{algorithm}

\subsubsection{Functions and variables description}
\begin{itemize}
	\item \textbf{map}: Variable referencing the map object created using the Google Maps API.
	\item \textbf{Map.getBounds()}: Method of the \textit{map} object that returns an object describing
                                  the boundaries of the currently visible map.
	\item \textbf{\textsc{getAvailableCars}(Bounds)}: Is a function that returns a list of car objects
                                                    containing all the currently available cars
                                                    inside the boundaries defined by the
                                                    \textit{Bounds} parameter.
	\item \textbf{List.size()}: Method that returns the number of elements contained in the list on which
                              it is called.
	\item \textbf{Car.getLatLng()}: Method that returns a variable containing the current latitude and
                                  longitude of the car on which it is called.
	\item \textbf{Car.getId()}: Method that returns the identification code of the car on which it is
                              called.
	\item \textbf{Bounds.contains(LatLng)}: This method returns true in the case in which the
                                          geographical position described by its argument
                                          \textit{LatLng}, is inside the boundaries defined by
                                          the \textit{bounds} object on which the method is called.
	\item \textbf{Map.addCar(CarMarker)}: This method adds the \textit{CarMarker} object that receives
                                        as a parameter to the map on which it's called, making the car
                                        and all its informations visible on the map, and allowing the
                                        user to reserve it.
	\item \textbf{CarMarker}: Object containing the all the necessary informations concerning the car.
	\item \textbf{Map.removeCar(CarId)}: Method that removes the car identified by the \textit{CarId}
                                       parameter from the map on which it is called.
\end{itemize}

\pagebreak

\subsection{Car reservation}
The following algorithm takes care of the reservation of a car by a user. In order to avoid consistency
issues we need to ensure the atomicity of the reservation so that a car cannot be reserved by more than
one user. To do so we decided to use a lock-like pattern.


\begin{algorithm}

  \algblock[TryCatchFinally]{try}{endtry}
  \algcblock[TryCatchFinally]{TryCatchFinally}{finally}{endtry}
  \algcblockdefx[TryCatchFinally]{TryCatchFinally}{catch}{endtry}
  	[1]{\textbf{catch} #1}
  	{\textbf{end try}}

\begin{algorithmic}[1]
  \caption{\label{alg:stdreqhandler} Car Reservation Handling Algorithm}

  \Function{carReservation}{carId, user}
    \State $ reservationLock = null $
    \State $ car = null $
    \try
      \State $ reservationLock = \Call {getReservationLock}{carId, user} $
      \State $ car = \Call {getCar}{carId} $
      \If {$ !car.isAvailable() $}
        \State {\textbf{Throw} carNotAvailableException}
      \EndIf
      \If {$ user.hasActiveReservation() $}
        \State {\textbf{Throw} multipleReservationException}
      \EndIf
      \State $ reservationTimer = Timer $(\textsc{reservation\textunderscore timer})
      \State \Call {createReservation}{carId, user, reservationTimer}{$ $}
      \State $ ReservationLock.releaseLock() $
    \catch {(LockTimeoutException, carNotAvailableException)}
      \State {\textbf{return} ReservationError}
    \catch{(CarNotFoundException, multipleReservationException)}
      \State $ reservationLock.releaseLock() $
      \State {\textbf{return} ReservationError}
    \endtry
    \State \textbf{return}
  \EndFunction

\end{algorithmic}
\end{algorithm}

\subsubsection{Functions and variables description}
\begin{itemize}
	\item \textbf{\textsc{getReservationLock}(CarId, User)}: Function that tries to acquire the lock on the
                                                           reservation part of the Car object and the
                                                           user object identified by the parameters
                                                           \textit{CarId} and \textit{User}.
                                                           If it can't get the lock after a certain
                                                           amount of time the function throws a
                                                           \textit{LockTimeoutException}.
	\item \textbf{\textsc{getCar}(CarId)}: Function that returns the Car object identified by the
                                         \textit{CarId} parameter.
                                         If no car corresponding to the specified ID is found, the
                                         function throws a \textit{CarNotFoundException}.
	\item \textbf{Car.isAvailable()}: Method that returns \textit{True} if the car on which it is called
                                    is currently available, returns \textit{False} instead.
	\item \textbf{Car.reserveCar()}: Method that set the state of the car on which it is called to
                                   \textit{Reserved}.
                                   This method is subject to the \textit{ReservationLock} obtainable
                                   through the \textit{\textsc{getCarReservationLock(CarId)}} function.
	\item \textbf{User.hasActiveReservation()}: Method that returns \textit{True} if the user on
                                                       which it is called on already has an active
                                                       reservation, returns \textit{False} instead.
	\item \textbf{ReservationTimer(minutes)}: This timer expires after the amount of minutes defined by the
                                            parameter \textit{minutes}, once expired it deletes the
                                            reservation it is associated to.
	\item \textbf{\textsc{reservation\textunderscore timer}}: Variable containing the duration of the
                                                            \textit{ReservationTimer} expressed in
                                                            minutes.
  \item \textbf{\textsc{createReservation}(CarId, User, ReservationTimer)}: Function that creates the
                                                                            reservation, associating the
                                                                            car, the user and the timer
                                                                            it gets as parameters.
                                                                            Among other things, it is
                                                                            in charge of setting the car
                                                                            as \textit{Reserved}, and
                                                                            starting the \textit{Timer}.
                                                                            The methods called in this
                                                                            function are subject to
                                                                            the \textit{ReservationLock}
                                                                            obtainable through the
                                                                            \textit{\textsc{getReservationLock}(CarId, User)}
                                                                            function.
	\item \textbf{Lock.releaseLock()}: Function that releases the lock on which it is called.
\end{itemize}

\pagebreak

\subsection{Selected architectural styles and patterns}

\label{sec:selected-styles-patterns}

In this section the main architectural styles and patterns are listed and described, motivating the reason why they have been chosen and how they integrate into the system architecture.

\begin{itemize}	
	\item \textbf{Client/Server architecture}: the mainly interaction paradigm between customers of the service and the software system is based on a \textit{request-response} model. Furthermore, interaction between Application tier and the Data tier work by means of the same paradigm. To support this kind of interaction the client/server architecture is chosen. Therefore, the client/server paradigm is inserted dividing the request logic from the response logic and placing them into distinct tiers. 
	\item \textbf{Distributed Representation}: in order to keep the client side as light as possible, a client/server paradigm based on distributed representation is chosen. All the application logic layer and the data layer are place on system side. The presentation layer, instead, is splittedin two part:
	\begin{itemize}
		\item \textbf{System side}: the presentation layer consist in software for the generation of responses, like web pages or meta documents.
		\item \textbf{Client side}: the presentatino layer consist in software for the interpretation of system's responses, such as web browser.
	\end{itemize}
	\item \textbf{4 Tiers}: the client/server architecture is split into 4 distinct tiers, decoupling data, application logic, request/response logic and client logic. In this way, the general maintenance of the system is dramatically improved, introducing an high modularity factor in the architecture.
	\item \textbf{Elastic Infrastructure}: the environment where the software system will be deployed is characterized by an unpredictable workload. To face optimaly the continuos and unpredictable change in requests load, an elastic infrastructure is introduced to support the whole system. It exposes a set of API is possible to get informations about resources utilizations and to perform different tasks, even automatized, reagarding provisioning and decommissioning of IT resources.
	\item \textbf{Elastic Platform}: to get the maximum from each IT resource available, the software environment hosted by each resource, such as an operating system, can be shared among multiple components, for instance multiple web servers or application components used by different users. To achieve this result, a software called  \textit{component manager} is used to manage the software components. It exposes an interface usable for provisioning and decommissioning of component instances.
	\item \textbf{Elastic Load Balancer}: to balance correctly the workload among the available resources, a load balancer is placed between the source of requests and the target. By the number of incoming requests and informations about the resources utilization, the load balancer distribute the work to the resources and can dynamically allocate more or less resources or application instances through the interface provided by the Elastic Infrastructure and the Elastic Platform.
\end{itemize}

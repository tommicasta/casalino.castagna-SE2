\subsection{Selected architectural styles and patterns}

\label{sec:selected-styles-patterns}

Here the main architectural styles and/or patterns are listed and described, motivating the reason why they have been chosen and how they integrate into the system.

\begin{itemize}	
	\item \textbf{Client/Server paradigm}: since most of the interaction between the system and the "outside" happens through requests and responses, the Client/Server paradigm result to be the most suitable design choice for the system.
The paradigm is implemented through the introduction of web servers components that have the task of receive the incoming requests, process them and answer back with responses messages.
	\item \textbf{Distributed Representation}: in order to keep the client side as thin as possible, a client/server paradigm based on distributed representation is found to be the best solution possible for the project. All the business logic, the data and part of the presentation layer are placed on the server side, while the client side is loaded with part of the presentation layer.
	\item \textbf{4 Tiers}: the client/server architecture is split into 4 distinct tiers, decoupling data, business logic, request/response logic and client logic. In this way, the general maintenance of the system is dramatically improved, introducing an high modularity factor in the architecture.
	\item \textbf{Elastic Load Balancer/Elastic Component}: this cloud pattern is introduced in order to improve the general availability of the system. It is placed in front of the Web Server Tier and through the incoming requests and through the runtime information regarding the workload of the servers the Load Balancer allocate the workload in a balanced fashion.
\end{itemize}

	\subsubsection{Utility tree}

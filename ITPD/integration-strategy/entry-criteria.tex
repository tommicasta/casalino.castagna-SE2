\subsection{Entry Criteria}

Before that the integration test phase of specific components may take place, it is fundamental that the following criteria (or conditions) here describe are met.
The verification of these conditions is really important in order to have as output of the integration test phase meaningful result, useful to assess the quality of the software system designed and, possibly, improve it.

It's worthful point out that some of the chosen criteria are strictly tied to the kind of strategy chosen to perform the integration testing of the system's components. To have informations about the integration testing strategy chosen for this software system, please refer to the \textit{\nameref{sec:integration-testing-strategy}} section.

First criteria to be met before any phase of the integration testing process may take place is the complete draw up of the \textbf{Requirements Analysis and Specification Document} and its positive assessment, since the RASD is the first and most important source of comparison that the development team has to check the testing results. This is more a general condition, refered to the whole testing process than to the specific component testing phase.

Since the integration testing follows a functional-integration based strategy, each core functionality implemented by each component, and supporting the software functionality to be tested, should be completely tested. The kind of test to perform should be both statical, i.e. through code inspections, and dynamical through the use of unit test cases. 

Before the integration test of a functionality may take place, the components supporting that functionality, their interfaces and their interaction should be described in the \textit{Design Document}, in order to have a overall view on how the components interacts and what kind of services they provide to support the main software functionality to be tested.

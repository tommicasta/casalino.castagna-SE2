\subsection{Integration Testing Strategy}
\label{sec:integration-testing-strategy}

Since the system to be developed is quite large and there's a quite number of components that can work independently from others, the most indicated strategy to approach the integration testing process is a blending of \textbf{thread-integration} method and \textbf{bottom-up} method.

To be more specific, the \textbf{thread-integration} method is \textit{functionality-oriented}, meaning that a system functionality at a time is tested. The testing of a system functionality is achieved by the development of specific components' parts that support that functionality (this is why the component-functionality mapping must be complete and correct) and integrating them together and testing. The use of this approach allows the developers to focus on one functionality at a time and to focus on the interaction between the involved components, easing the bugs discovery, interaction issues and performance issues.

The \textbf{bottom-up} approach, instead, consists in the development of complete components and integration of them. In this project this approach is used for the integration of atomic components, that is either components which functionalities are independent from other components or components already developed, such as external components. The main benefits brought by this strategy are the simplification of the scaffolding plan thanks to the reduction of needed stubs (real components can be used for the testing), speeding up the integration process (real components are integrated from the beginning), ease the bugs discovery and interaction/performance issues discovery.

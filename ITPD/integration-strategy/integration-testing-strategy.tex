\subsection{Integration Testing Strategy}
\label{sec:integration-testing-strategy}

The architecture designed for the system software under development [DD 2.2] [DD 2.3] is characterized by a \textbf{small extension}, namely a restricted number of components, and by a \textbf{loose coupling factor}, due to the presence of several indipendent components.

Making reference to the integration test strategies present in software engineering literature, the category of \textbf{structural integration strategies} results to be the most indicated for small and medium projects.
Other integration categories, like the functional-oriented strategies, require a planning effort that overcomes the output of the process, making them useless.

Among the strategies belonging to the structural category, the \textbf{Bottom-up strategy} proves to be the most appropriated approach, exploiting at the best the simplicity of the system's architecture, avoiding complex and unnecessary scaffolding, and the presence of low level indipendent components.

Starting from the bottom of the architecture's hierarchy, the process integrates step by step the indipendent components into small subsystems, shifting afterwards to integrate the small subsystems obtained from the previous steps, climbing up the hierarchy until all the components are integrated and tested.

Each step of the integration testing activity is supported by the implementation of \textit{drivers}, which emulate the behaviour and the calls performed by one higher-level components towards the lower components to which it is linked, allowing the integration and testing of the architecture.
Once one of the small subsystems is thoroughly tested and integrated, the driver can be substituted by the real implementation of the component emulated.

\subsection{Integration Testing Strategy}
\label{sec:integration-testing-strategy}

The system's architecture design, presented in [DD 2.2] and described in more detail in [DD 2.3], shown a relatively low architecture complexity. Functionality-oriented strategies, i.e. thread strategy, or critical-modules-oriented strategies are not considered appropriate since the planning efforts needed exceed the possible outcomes derived from the use of these strategies.

Structural-oriented strategies, instead, shown to be quite simply to plan and apply. Among the possible strategies of this testing category, the one considered most indicated to approach the integration test activity is the \textbf{Top-Down Strategy}.

The choice is driven by three main features that characterize this testing approach:

\begin{itemize}
	\item \textbf{Unit test prioritization:} after a component is implemented, it must be immediately tested before any other activity, i.e. implementation or testing another component, can be performed. This means that possible implementation errors, like functionality discrepancies or performance issues, may be discovered immediately after the implementation phase.
	\item \textbf{Ease design flaws detection:} structural errors in the software system's architecture can be easier detected, avoiding extensive design re-implementation or correction.
	\item \textbf{Early availability of a prototype:} the presence of a prototype allows an early validation stage. This means that eventual discrepancies between what has been designed in terms of functionalities and what has been developed may be detected in due time.
\end{itemize}

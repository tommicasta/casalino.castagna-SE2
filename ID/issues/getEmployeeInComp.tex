\subsection{\textcolor{blue}{getEmployeeInComp} Method}

\begin{enumerate}
	\item \textbf{C1} The method's name is quite misleading since it does not suggest that the function returns informations about currently active employee positions authorized by the party group identified by the \textcolor{blue}{partyId} passed as parameter.

	\item \textbf{C1} \textcolor{blue}{isEmpls} variable's name does not suggest to store the list of employee positions authorized by the party group.

	\item \textbf{C1} \textcolor{blue}{childOfEmpl} variable's name does not suggest to store the current employee position visited through the \textcolor{blue}{for} cycle at \textbf{L.269}.

	\item \textbf{C1} \textcolor{blue}{emlpfCtxs} variable's name does not suggest to store the list of fulfillment for a certain employee position.

	\item \textbf{C1} \textcolor{blue}{emplContext} variable's name does not suggest to store a \textcolor{blue}{GenericValue} object initialized with the informations concerning the \textcolor{blue}{EmplPositionType} instance related to the employee position currently analyzed through the for cycle at \textbf{L.269}.

	\item \textbf{C13} At \textbf{L.263}, the \textcolor{blue}{EntityCondition.makeCondition()} method invocation exceeds the suggested maximum length of 80 characters per line. Possible to break after the first comma.

	\item \textbf{C13} At \textbf{L.264}, the \textcolor{blue}{EntityCondition.makeCondition()} method invocation exceeds the suggested maximum length of 80 characters per line. Possible to break after the first comma.

	\item \textbf{C13} At \textbf{L.279}, the statement exceeds the suggested maximum length of 80 characters per line. Possible to break after the second chain operator.

	\item \textbf{C13} At \textbf{L.289}, the statement exceeds the suggested maximum lenght of 80 characters per line. Possible to break after the second "+" operator.

	\item \textbf{C13} At \textbf{L.294}, the statement exceeds the suggested maximum lenght of 80 characters per line. Possible to break after the second "+" operator.

	\item \textbf{C14} The method signature at \textbf{L.254} exceeds the maximum legal length of 120 characters per line. Possible to break it before the \textcolor{blue}{throws} keyword.

	\item \textbf{C14} The statement at \textbf{L.286} exceeds the maximum legal length of 120 characters per line. Possible to break it after the second chain operator.

	\textbf{NB:} applying this correction another \textbf{C14} issue is created at \textbf{L.287}, where the rest of the statement at \textbf{L.286} is placed. To solve it is possible to break after the first chain operator at \textbf{L.287}.

	\item \textbf{C29} At \textbf{L.255}, the variable
		\textcolor{blue}{isEmpls} can be declared inside the following
		\textcolor{blue}{try} block since it is the only place where the variable is
		used. Doing so we can directly initialize the variable with the proper value
		instead of \textbf{null}.

	\item \textbf{C33} Variables at \textbf{L.286-287} are not declared at the beginning of the for block at \textbf{L.269}.

	\item \textbf{C52} The \textcolor{blue}{EntityQuery.queryOne()} method invoked at \textbf{L.286} may throw an \textcolor{blue}{IllegalArgumentException} when the list passed to the \textcolor{blue}{EntityUtil.getOnly()} method has more than one argument.
		None of the previously mentioned methods, nor the method under inspection, specify this expection in a \textbf{try/catch} block.

	\item \textbf{C52} \textcolor{blue}{EntityCondition.makeCondition()} methods invoked at \textbf{L.263-264} may throw an \textcolor{blue}{IllegalArgumentException} due to the \textcolor{blue}{EntityExpr} constructor called by the \textcolor{blue}{makeCondition} method itself. No \textbf{try/catch} block managed this exception and no \textcolor{blue}{throws} is used to pass the exception to higher levels.

	\item \textbf{C53} The \textcolor{blue}{Debug.logError()} method does not log any useful information about the error.
		Indeed, the logError calls the \textcolor{blue}{Debug.log()} method, which has a \textbf{"null"} value as message.

		Then it calls another version of \textcolor{blue}{Debug.log()}, which simply get a logger and stores the level danger of the error, the msg, which is \textbf{"null"} and the informations about the exception.
\end{enumerate}

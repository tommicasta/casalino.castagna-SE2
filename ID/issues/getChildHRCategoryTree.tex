\subsection{\textcolor{blue}{getChildHRCategoryTree} Method}

\begin{enumerate}

  \item \textbf{C10} Declaration of method \textcolor{blue}{getChildHRCategoryTree}
    has an inconsistent bracing style. A space before the first brace is
    missing, unlike all the other opening blocks.

  \item \textbf{C13} There are many lines that exceed the
    character limit: \textbf{L.41}, \textbf{L.57} and \textbf{L.68}

    \begin{itemize}

      \item At \textbf{L.41} the signature of the method
        \textcolor{blue}{getChildHRCategoryTree} can be split after the first
        parameter though it's not very readble.

      \item At \textbf{L.57} the declaration of the variable
        \textcolor{blue}{categoryList} exceeeds the 80 character libit only by
        4, it could be split after the \textbf{=} operator. Due to the fact that
        the character limit is exceeded only by 4 characters, this separation
        could be avoided to benefit readability.

      \item At \textbf{L.68} the chain method invocation can be split before the
        \textcolor{blue}{from()} method is called.

    \end{itemize}

  \item \textbf{27} The \textcolor{blue}{try-catch} blocks at \textbf{L.62} and
    at \textbf{L.77} have the same catch block, they could be merged into one.

  \item \textbf{C33} The declaration of the variable
    \textcolor{blue}{List<Map<String,Object>> categoryList} at \textbf{L.57}
    does not appear at the beginning of the block.

  \item \textbf{C40} At \textbf{L.69} the object \textcolor{blue}{partyGroup} is
    compared with \textbf{null} using the \textbf{!=} operator instead of the
    \textcolor{blue}{isNotEmpty()} helper function.

  \item \textbf{C42} At \textbf{L.62} and \textbf{L.77} the \textcolor{blue}{catch}
    blocks only provide the stack trace when catching an exception, not providing
    any guidance on how to correct the problem.

  \item \textbf{C52} The \textcolor{blue}{EntityQuery.queryOne()} method invoked
    at L.68 may throw an \textit{IllegalArgumentException} when the list passed
    to the \textcolor{blue}{EntityUtil.getOnly()} method has more than one
    argument. None of the previously mentioned methods, nor the method under
    inspection, specify this expection in a \textbf{try/catch} block.

\end{enumerate}

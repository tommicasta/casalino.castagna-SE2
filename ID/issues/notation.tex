\subsection{Notation}

During the code inspection of the class and the methods, some notations are used to ease the reporting:

\begin{itemize}
	\item To make reference to a certain source line code, the \textbf{L.value} notation is used. 

Example: \textit{L.123 referes to line 123 of the class.}
	\item To make reference to a block source lines code, the \textbf{L.val1-valN} notation is used. 

Example: \textit{L.123-456 referes to the block of source lines of code starting from line 123 and ending at line 456.}
	\item To make reference to a specific issue of the check list, the \textbf{Cnumber} notation is used.

Example: \textit{C42 referes to the 42th item of the checklist.}
\end{itemize}

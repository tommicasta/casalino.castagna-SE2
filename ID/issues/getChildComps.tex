\subsection{\textcolor{blue}{getEmployeeInComp} Method}

\begin{enumerate}

  \item \textbf{C1} The variable \textcolor{blue}{catId} at \textbf{L.180} has a
    name that does not suggest that it contains the ID of the party in which is
    contained the node that we're inspecting.

  \item \textbf{C1} The variable \textcolor{blue}{catNameField} at
    \textbf{L.181} has a name that does not suggest that it contains the
    name of the \textit{PartyGroup} related to the node we're inspecting.

  \item \textbf{C1} The variable \textcolor{blue}{childContext} at
    \textbf{L.192} has a name that does not suggest that it contains the
    information of the \textit{PartyGroup} related to the node we're inspecting.

  \item \textbf{C1} The variable \textcolor{blue}{childOfSubComs} at
    \textbf{L.200} has a name that does not immediatly suggest that it contains
    the list of child nodes related to the node we're inspecting.

  \item \textbf{C1} The variable \textcolor{blue}{isPosition} at
    \textbf{L.205} has a name that does not immediatly suggest that is used to
    check if the node that we are inspecting has an employment position related
    to it.

  \item \textbf{C1} The variable \textcolor{blue}{emContext} at
    \textbf{L.211} has a name that does not immediatly suggest that it contains
    the information of an eployee related to the node we are inspecting, instead
    it could be named \textcolor{blue}{employeeContext} to make things more
    clear.

  \item \textbf{C10} Declaration of method \textcolor{blue}{getChildComps} has
    an inconsistent bracing style. A space before the first brace is missing,
    unlike all the other opening blocks.

  \item \textbf{C13} There are many lines that exceed the
    character limit: \textbf{L.160}, \textbf{L.168}, \textbf{L.169},
    \textbf{L.184}, \textbf{L.186}, \textbf{L.192}, \textbf{L.200},
    \textbf{L.205}, \textbf{L.206}, \textbf{L.211} and \textbf{L.224}

    \begin{itemize}

      \item At \textbf{L.60} the signature of the method
        \textcolor{blue}{getChildComps} can be split after the first parameter
        to srictly maintain the 80 character limit though it's not very
        readble. A better solution that maintain readability could be to split
        before the \textbf{throws} keyword and exceeding the limit by only 6
        characters.

      \item At \textbf{L.168} the declaration of the variable
        \textcolor{blue}{partyGroup} exceeeds the 80 character libit only by
        8, it could be split after the \textbf{=} operator. Due to the fact
        that the character limit is exceeded only by a small amount, this
        separation could be avoided.

      \item At \textbf{L.169} the declaration of the variable
        \textcolor{blue}{resultList} exceeeds the 80 character libit only by
        3, it could be split after the \textbf{=} operator. Due to the fact
        that the character limit is exceeded only by a small amount, this
        separation could be avoided.

      \item At \textbf{L.184} the declaration of the variable
        \textcolor{blue}{josonMap} exceeeds the 80 character libit only by
        1, it could be split after the \textbf{=} operator. Due to the fact
        that the character limit is exceeded only by a small amount, this
        separation could be avoided.

      \item At \textbf{L.186} the declaration of the variable
        \textcolor{blue}{dataAttrMap} exceeeds the 80 character libit only by
        4, it could be split after the \textbf{=} operator. Due to the fact
        that the character limit is exceeded only by a small amount, this
        separation could be avoided.

      \item At \textbf{L.192} the chain method invocation can be split before
        the \textcolor{blue}{from()} method is called.

      \item At \textbf{L.200} the line could be split after the \textbf{=}
        operator to follow the 80 characters limit, though losing some
        readability.

      \item At \textbf{L.205} the chain method invocation can be split before
        the \textcolor{blue}{from()} method is called.

      \item At \textbf{L.206} the concatenated conditions could be split after
        the \textbf{||} operator.

      \item At \textbf{L.211} the chain method invocation can be split before
        the \textcolor{blue}{from()} method is called.

      \item At \textbf{L.224} line could be split after the comma.

    \end{itemize}

  \item \textbf{C29} At \textbf{L.168} and \textbf{L.170}, the variables
    \textcolor{blue}{partyGroup} \textcolor{blue}{childOfComs} can be declared
    inside the following \textcolor{blue}{try} block since it is the only place
    where the variables are used. Doing so we can directly initialize the
    variables with the proper value instead of \textbf{null}.

  \item \textbf{C29} At \textbf{L.181}, the variable
    \textcolor{blue}{catNameField} can be declared inside the following
    \textcolor{blue}{if} block since it is the only place where the variable is
    used. Doing so we can directly initialize the variable with the proper value
    instead of \textbf{null}.

  \item \textbf{C33} The declarations at \textbf{L.184-187}, \textbf{L.192},
    \textbf{L.200}, \textbf{L.205}, \textbf{L.211} and \textbf{L.226} are not
    made at the beginning of the block.

  \item \textbf{C42} At \textbf{L.245} the \textcolor{blue}{catch} block only
    provides the stack trace when catching an exception, not providing any
    guidance on how to correct the problem.

  \item \textbf{C52} The \textcolor{blue}{EntityQuery.queryOne()} method invoked
    at \textbf{L.192} and \textbf{L.211} may throw an
    \textit{IllegalArgumentException} when the list passed to the
    \textcolor{blue}{EntityUtil.getOnly()} method has more than one argument.
    None of the previously mentioned methods, nor the method under inspection,
    specify this expection in a \textbf{try/catch} block.

  \item \textbf{53} The \textcolor{blue}{catch} block at \textbf{L.245}
    generates a new exception of the same type of the one caught, without
    adding any information to it. A better solution could be to throw the same
    exception caught, or to decorate the new exception with more informations.
\end{enumerate}
